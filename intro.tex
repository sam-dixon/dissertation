\chapter{Introduction}

\section{Expansion History of the Universe}
Cosmology is the study of the origin, evolution, and eventual fate of the Universe. The current standard model of cosmology, the LCDM \ model, posits that the Universe is governed by Einstein's equations of relativity with a cosmological constant $\textrm{Lambda}$ and cold dark matter. 

Einstein's field equations (including a cosmological constant) are
\begin{equation}
    G_{\mu\nu} + \Lambda g_{\mu\nu} = \frac{8\pi G}{c^4}T_{\mu\nu}
    \label{eq:einstein_field}
\end{equation}
Under this model, we use the Friedmann-Lema\^{i}tre-Robertson-Walker (FLRW) metric to encode the homogeneous, isotropic, and time-dependent expansion of space:
\begin{equation}
    -c^2 d\tau^2 = -c^2dt^2 + a(t)^2\left[dr^2 + S_k(r)d\Omega^2\right]
    \label{eq:flrw_metric}
\end{equation}
where
\begin{equation}
    S_k(r) =
    \begin{cases}
    R\sin(r/R) & k=+1\\
    r & k=0\\
    R\sinh(r/R) & k=-1
    \end{cases}
    \label{eq:curvature}
\end{equation}

All of the time-dependence is captured by the evolution of the scale factor $a$. With this metric, we can find the following analytic solution to Einstein's field equations:
\begin{align}
    \left(\frac{\dot{a}}{a}\right)^2 & = \frac{8\pi G}{3} \rho - \frac{kc^2}{a^2} + \frac{\Lambda c^2}{3}\\
    \frac{\ddot{a}}{a} & = -\frac{4\pi G}{3}\left(\rho + \frac{3p}{c^2}\right) + \frac{\Lambda c^2}{3}
    \label{eqn:friedmann_eqns}
\end{align}
where $k$ is the curvature, $\rho$ is the density and $p$ is the pressure. These are known as the Friedmann equations. The Hubble parameter $H$ is defined at By defining a critical density
\begin{equation}
    \rho_c = \frac{3H^2}{8\pi G}
\end{equation}
we can then 

\section{Supernova Cosmology}
The focus of the studies presented here are on improving our understanding of Type Ia supernovae as tools for measuring cosmological distances. Type Ia supernovae were believed to 

\section{SNIFS: the SuperNova Integral Field Spectrograph}
The vast majority of the data analyzed in this thesis was collected using the SuperNova Integral Field Spectrograph (SNIFS) by members of the Nearby Supernova Factory (SNfactory) collaboration. SNIFS is mounted 

\section{Empirical Models of Type Ia Supernovae}

\section{Summary}
