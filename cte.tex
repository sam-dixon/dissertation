\chapter{Measurements of Charge Transfer Efficiency in the SNIFS Detectors}
\label{chap:cte}

\section{Introduction}
Charge-coupled devices (CCDs) are ubiquitous in optical and infrared astronomy. Incident photons generate electrons in the bulk silicon of the CCD, and these photoelectrons are collected by a grid of pixels arranged on a series of parallel readout registers. By manipulating the voltages along the gates of these registers, we can shift the collected charge row-by-row into a second (serial) register. Similar voltage manipulations of the serial register create a signal that can be amplified and digitized. The charge shuffling alternates between the parallel and serial registers until all pixels have been read out and the image is reconstructed in software from the resulting signal.

Ideally, 100\% of the charge in each pixel would be transferred to the next pixel at each readout step. However, traps in the silicon lattice of the CCD can capture charge only to release it at a later time. The trapped and released charge leads to smearing of point sources along the direction of charge transfer and impedes our ability to get accurate photon counts. The magnitude of this effect is quantified by the charge transfer efficiency (CTE) of the CCD, defined as the fraction of charge that survives each pixel transfer. Because most of the CCDs used in astronomical applications make many transfers in each image (as they have thousands of pixels in each register), the CTE must be very close to unity, with typical values being around $1-10^{-6}$. Additionally, because a larger number of transfers results in a higher likelihood of encountering more traps, pixels that are further from the readout register are more effected by charge trapping. This effect is especially problematic for spectroscopic instruments since the increased smearing with distance to the amplifier can lead to effects like uneven broadening of spectral features.

In this work, we quantify the charge transfer efficiency of the CCDs in all channels (photometric, and blue and red spectroscopic) of the SuperNova Integral Field Spectrograph (SNIFS) using two methods. The initial characterization was done by first exposing the CCDs to a Fe-55 x-ray source in order to generate single-pixel events with a known energy spectrum and then measuring the photoelectron loss. Later, we performed a similar analysis using cosmic ray events extracted from the dark frames taken during each observing run, enabling in-situ measurements of the CTE.

\section{Initial Characterization with Fe-55 X-rays}
The standard method for measuring CTE involves exposing the CCD to photons with a known spectrum \cite{janesick_scientific_2001}. These photons generate single-pixel events in the CCD image with a known amount of charge. By plotting the amount of charge measured in each event as a function of the row number, we can quantify the extent of charge trapping by measuring the decrease in charge with distance from the amplifier. A common source for these characterizations is Fe-55, which produces strong lines from K-alpha and K-beta emission at 5.9 keV and 6.2 keV. These emission lines corresponding to photoelectron signals in the CCD of 1620 $\textrm{e}^-$ and 1778 $\textrm{e}^-$, respectively.

 We performed this characterization by exposing each of the SNIFS CCDs to a Fe-55 source. Using \verb|sep|, a Python implementation of the commonly-used \verb|SExtractor| package, we selected all events in the resulting images that had signal levels $>5\sigma$ above the background signal level. We then fit the histogram of digitized counts in these extracted events, $s(f)$, to a model of the Fe-55 spectrum:
\begin{equation}
    s = A_1 \exp\left(-\frac{(f-1620/g)^2}{2\sigma_1^2}\right) + A_2 \exp\left(-\frac{(f-1778/g)^2}{2\sigma_2^2}\right).
\end{equation}
This fit is done in order to measure the gain, $g$, of the CCD amplifier, allowing us to convert from ADU counts to electrons. The relative fraction of events from the K-alpha and K-beta emission (i.e. $A_1/A_2$), as well as the width of the emission lines are allowed to float, treating these quantities as nuisance parameters in the fit.

\begin{figure}
    \centering
    \includegraphics{figures/cte/spectrum_fit.pdf}
    \caption{Example spectrum of the Fe-55 X-rays events extracted from a single amplifier frame along with the best-fit model. The results of this fit are used to determine the amplifier gain from each frame.}
    \label{fig:xray_spectrum}
\end{figure}

We combined the events from several images per channel (22 for the blue channel, 26 for the red channel, and 8 for the photometric channel), using the gain measured from the spectrum fit for each image to convert the flux from ADU counts to number of electrons $N_{e^-}$. We select the events that correspond to K-alpha emission by choosing events with $1550 < N_{e^-} < 1700$. We then plot the flux for each event as a function of distance from the amplifier, shown in Fig. \ref{fig:cte_xray} for the parallel registers and Fig. \ref{fig:cte_xray_serial} for the serial register. The slope of the best-fit line (in units of $e^-$ per transfer), divided by the expected number of electrons (1620), gives us our measurement of the charge transfer inefficiency, or what fraction of charge is lost in each transfer. The resulting measurements of the charge transfer inefficiency are shown as labels in Figs. \ref{fig:cte_xray} and \ref{fig:cte_xray_serial}, as well as in Table \ref{tab:cte_xray}.

\begin{figure}[htbp]
    \centering
    \includegraphics[width=0.85\textwidth]{figures/cte/xray_cte_parallel.png}
    \caption{The number of electrons in extracted Fe-55 X-ray events as a function of y-location on the CCD for each camera amplifier (equivalent to number of transfers in the parallel direction). The slope of the best-fit line divided by the expected number of electrons gives us an estimate of the charge transfer inefficiency in the parallel registers.}
    \label{fig:cte_xray}
\end{figure}

\begin{figure}[htbp]
    \centering
    \includegraphics[width=0.85\textwidth]{figures/cte/xray_cte_serial.png}
    \caption{Same as Fig. \ref{fig:cte_xray}, but for the x-location of events. The best-fit lines are used to estimate the charge transfer inefficiency in the serial registers.}
    \label{fig:cte_xray_serial}
\end{figure}

\begin{table}[htbp]
    \centering
    \begin{tabular}{cccccc}\toprule
        Camera & Amp. & Parallel CTI & Serial CTI & Number of frames & Number of events \\\midrule
        B & A &0.783 $\;\pm\;$ 0.015 & 6.80 $\;\pm\;$ 0.03 & 22 & 15,925 \\
          & B &1.519 $\;\pm\;$ 0.017 & 3.14 $\;\pm\;$ 0.03 &    & 13,113 \\\midrule
        R & A &2.312 $\;\pm\;$ 0.016 & 11.60 $\;\pm\;$ 0.04 & 26 & 4,071 \\
          & B &0.451 $\;\pm\;$ 0.011 & 4.26 $\;\pm\;$ 0.03 &    & 4,415 \\\midrule
        P & A &2.696 $\;\pm\;$ 0.010 & 1.33 $\;\pm\;$ 0.02 & 8 & 15,308 \\
          & B &0.773 $\;\pm\;$ 0.008 & 0.93 $\;\pm\;$ 0.01 &   & 21,010 \\
          & C &1.109 $\;\pm\;$ 0.009 & 3.48 $\;\pm\;$ 0.02 &   & 14,498 \\
          & D &0.160 $\;\pm\;$ 0.008 & 3.66 $\;\pm\;$ 0.01 &   & 19,041 \\\midrule
    \end{tabular}
    \caption{Charge transfer inefficiency results from Fe-55 X-ray characterization. All values are in units of $10^{-6}$.}
    \label{tab:cte_xray}
\end{table}

\section{Cosmic Ray Measurement}
The X-ray measurement allows for a precise determination of the CTI when we have physical access to the detector. However, we'd like to be able to track changes in the CTI with time \emph{in situ} in order to quantify any degradation over the lifetime of the instrument. We can get such a measurement by making use of the cosmic ray hits found in the dark frames taken as part of normal observing setup and measuring the smearing of these hits due to charge transfer inefficiency, similar to the methodology presented in \cite{riess_time_1999}.

Cosmic rays create small (approximate 6-7 pixels) events in digitized CCD images. The shape of each individual event is driven by the incidence angle of the cosmic ray and charge diffusion (i.e. bleeding between pixels because of the structure of the CCD), but also by the CTI smearing effect. The first two effects are statistically symmetric about the highest pixel, so in principle we should be able to subtract the symmetric portions of the cosmic ray events, leaving only the asymmetric CTI trails.

We proceed very similarly to \cite{riess_time_1999}. For each dark frame, we use \verb|sep| to find cosmic ray hits, defined as the objects detected at $>1.5\sigma$ over the background level, with a measured ellipticity $< 0.2$ and no flags raised by \verb|sep|. The ellipticity cut serves to remove extremely oblique incidence events or coincident events from our sample, as both types of event add excess noise to the measurement. Example hits that pass these cuts are shown in Fig. \ref{fig:example_hits}. Additionally, in order to avoid the noise potentially introduced by hits landing in the CTI trails of other nearby hits (see e.g. the bottom right example in Fig. \ref{fig:example_hits}), we remove from our sample all pairs of events that are within 10 pixels of one another.

\begin{figure}
    \centering
    \includegraphics[width=0.9\textwidth]{figures/cte/example_hits.png}
    \caption{Example identified cosmic ray hits passing our ellipticity cut. The two neighboring hits seen in the bottom right example would be removed from the final sample because they are too close to each other. This culling removes some of the noise from our final signal.}
    \label{fig:example_hits}
\end{figure}

For each selected cosmic ray event, we subtract the value of the pixels further from the readout amplifier from the pixels closer to the readout amplifier in both the parallel and serial direction. We calculate the fraction of charge that is left in the trail by summing the number of excess counts in the 5 trailing pixels and dividing that sum by the number of counts in the peak of the event. On average, this should give us an estimate of the fraction of charge lost to trapping. 
    
We combine all of these measured charge loss fractions from every dark frame taken on a single night. The individual values of the charge loss fraction are shown in blue in Figs. \ref{fig:cte_single_night} and \ref{fig:cte_single_night_serial}. These measurements are extremely noisy, so to reduce the noise, we group them in 16 bins along the direction of interest and take the median and normalized median absolute deviation (NMAD) of all of the measurements in each bin. These median values are shown in red in Figs. \ref{fig:cte_single_night} and \ref{fig:cte_single_night_serial}, as well as in Fig. \ref{fig:cte_single_night_medians} and \ref{fig:cte_single_night_serial_medians} in more detail. Finally, we fit a line to these median values weighted by the NMAD in each bin. As before, the slope of the line gives us a measure of the charge transfer inefficiency.

\begin{figure}
    \centering
    \includegraphics[width=0.9\textwidth]{figures/cte/single_night_example_parallel.png}
    \caption{Example measurement of CTI in the parallel register from cosmic ray trails from a single night. All blue dots represent a single cosmic ray hit. The red points show the median fraction of counts in the peak of the hits to end up in the trails in each y-position bin. The best-fit line is also shown. The slope of this line gives us the CTI.}
    \label{fig:cte_single_night}
\end{figure}

\begin{figure}
    \centering
    \includegraphics[width=0.9\textwidth]{figures/cte/single_night_example_serial.png}
    \caption{Same as Fig. \ref{fig:cte_single_night} but in the serial direction.}
    \label{fig:cte_single_night_serial}
\end{figure}

\begin{figure}
    \centering
    \includegraphics[width=0.9\textwidth]{figures/cte/single_night_example_parallel_medians.png}
    \caption{Same as Fig. \ref{fig:cte_single_night} but zoomed to show the median values and their associated uncertainties.}
    \label{fig:cte_single_night_medians}
\end{figure}

\begin{figure}
    \centering
    \includegraphics[width=0.9\textwidth]{figures/cte/single_night_example_serial_medians.png}
    \caption{Same as Fig. \ref{fig:cte_single_night_medians} but in the serial direction.}
    \label{fig:cte_single_night_serial_medians}
\end{figure}

We made these measurements for every night's dark frames in order to check for time dependence. In Fig. \ref{fig:time_variation} we show the CTI in the parallel and serial registers as a function of time for each amplifier. Fig. \ref{fig:time_variation_binned} shows the same data aggregated by year. There is very little evidence of significant degradation over time. Indeed, linear fits to each of these data sets have slopes consistent with zero. Finally, the median CTI measured in each amplifier using cosmic rays in dark frames are all summarized in Table \ref{tab:cte_darks}.

\begin{figure}
    \centering
    \includegraphics[width=0.9\textwidth]{figures/cte/time_variation.png}
    \caption{A search for potential time variation in the charge transfer efficiency of each of the spectroscopic cameras. CTI measurements in the parallel direction are shown in blue and those in the serial direction are shown in orange.}
    \label{fig:time_variation}
\end{figure}

\begin{figure}
    \centering
    \includegraphics[width=0.9\textwidth]{figures/cte/time_variation_by_year.png}
    \caption{Same as Fig \ref{fig:time_variation} but aggregated by year.}
    \label{fig:time_variation_binned}
\end{figure}

\begin{table}[htbp]
    \centering
    \begin{tabular}{cccc}\toprule
        Camera & Amplifier & Median Parallel CTI  & Median Serial CTI \\ \midrule
        B & A &   1.08 $\pm$ 0.04  &  0.96 $\pm$ 0.13 \\
          & B &   0.97 $\pm$ 0.04  &  2.02 $\pm$ 0.13 \\\midrule
        R & A &   1.49 $\pm$ 0.07  &  2.4  $\pm$ 0.2 \\
          & B &   1.24 $\pm$ 0.07  &  2.3  $\pm$ 0.2 \\\midrule
        P & A &   0.19 $\pm$ 0.18  &  5.1  $\pm$ 0.5 \\
          & B &   0.18 $\pm$ 0.19  &  7.0  $\pm$ 0.4 \\
          & C &   0.3  $\pm$ 0.2   &  5.4  $\pm$ 0.7 \\
          & D &   0.5  $\pm$ 0.2   &  1.9  $\pm$ 0.5 \\\midrule
    \end{tabular}
    \caption{CTI measurements over time from all dark frames collected from 2006 to 2012. All values are in units of $10^{-6}$.}
    \label{tab:cte_darks}
\end{table}

\section{Conclusion}


