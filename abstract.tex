\begin{abstract}
Type Ia supernovae (SNe Ia) remain one of the best astronomical tools available for measuring large distances. Observations of SNe Ia were instrumental in the discovery of the accelerating expansion of the universe, and will be key to understanding the nature of the ``dark energy" driving this accelerating expansion. Surveys like LSST and the Roman Space Telescope supernova survey are currently being designed to discover and follow a large number of SNe Ia -- a large enough number that the statistical error will be far subdominant to the systematic error. In order to avoid biases in the cosmological parameters constrained by these future surveys, it is essential that we understand the sources and effects of systematic uncertainties. This dissertation addresses some of these systematic biases. 

In Chapter 2, we present two measurements of the extent of charge transfer inefficiency (CTI) in the detectors of the SNIFS instrument that was used to collect much of the data used throughout this dissertation. If the inefficiency is too high in spectroscopic instruments like SNIFS, the smearing that CTI causes can lead to misinterpretation of the resulting spectra. We find that the CTI is low in all detectors (about 1 photoelectron in every million is trapped), and that the low CTI remains stable over time.

Chapter 3 focuses on modeling a particular spectral region of Type Ia supernova spectra near maximum brightness. This region of the spectrum is used in some subclassification schemes of SNe Ia, and can also serve as a proxy for identifying changes in the populations of these subtypes with redshift. Our model can provide accurate measurements of ejecta velocities and the feature equivalent width using low-resolution and/or noisy spectra. Being able to use lower-quality spectra allows us to mitigate bias out to earlier eras of cosmic history by allowing us to monitor population drifts at higher redshifts.

In Chapter 4, we study two empirical models of Type Ia supernova spectral evolution (SALT2 and SNEMO) and measure how well they can capture a variety of near-maximum spectroscopic features. Our goal is to analyse how linear spectral models with differing number of parameters can capture non-linear features like ejecta velocities. In addition, we present a model for producing realistic mock spectra based on these models, allowing future studies to have access to spectral templates that capture the full range of supernova spectral behavior.

Chapter 5 centers on an assumption of linear regression that is often overlooked in supernova cosmology analyses. These analyses perform an initial linear regression to correct the observed SN absolute magnitudes for other properties of the SN. They then perform a second regression to correct the residuals of this first regression for an additional covariate. This practice is statistically sound only if the covariates in the initial regression are not correlated with the covariates used in the second regression. However, these correlations do exist. We present a toy model of this problem to calculate closed-form expressions and scaling relations of the size of the biases in the effect sizes and estimated scatter that come from this overlooked assumption. We also use simulations based on literature data to calculate the size of these biases and provide potential corrections.

Chapter 6 presents two new models of Type Ia supernova spectroscopy that were constructed using deep learning. These models extend the ``twins embedding" model of Boone et al. (submitted) into a wide range of phases. The \texttt{spec2embed} model takes as input a spectrum observed at any phase from -10 to +40 days after maximum brightness, and predicts the spectrum's phase and its supernova's location in the twins embedding space. Using these predictions, we can standardize supernovae from single spectra with comparable precision to the original twins embedding work. The \texttt{embed2spec} model works in reverse, taking a phase (or range of phases) and location in the twins embedding space to predict a spectrum. With this, we can use forward-modeling fitting techniques to constrain a supernova's location in the twins embedding space from multiple spectra, spectra with lower spectral resolution, or even broadband photometry.
\end{abstract}