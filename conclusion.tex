\chapter{Conclusions}

We have presented a number of tools for finding and measuring potential systemic errors and biases throughout the process of making cosmological measurements using Type Ia supernovae, from detection through standardization.

In Chapter 2, we presented measurements of the charge transfer inefficiency of CCD detectors used to collect much of the data used throughout the rest of this dissertation. The laboratory measurements provide a baseline for future analyses, and the in-situ measurements allow for tracking the charge transfer efficiency over time. We found that the charge transfer inefficiency was stable over time, and small enough to have little impact on further measurements.

Chapter 3 presented a model of the \ion{Si}{ii} $\lambda$5972 and \ion{Si}{ii} $\lambda$6355 region of Type Ia supernova spectra. This model was able to accurately extract measurements of this feature from low-resolution and noisy spectra, enabling the use of spectra from prism spectroscopy supernova surveys like that planned for the Roman Space Telescope. Since this spectral region can act as a probe of potential population drifts, this model provides a tool for monitoring these drifts out to higher redshifts.

We investigated a similar problem in Chapter 4, measuring how well a number of the inherently non-linear near-maximum brightness spectral features could be measured using SALT2 and SNEMO, two linear empirical models of Type Ia supernova spectral evolution. We found that, in general, linear spectral models with more parameters can more accurately capture these non-linear features. In this Chapter we also constructed a model for producing realistic fake spectra using kernel density estimation of the latent parameter space of these models. This more flexible latent space model was better able to mimic the distributions of the spectral features than simpler (e.g. multivariate Gaussian) models of these latent spaces.

We addressed a generalized statistical problem in Chapter 5, as it appears in a number of Type Ia supernova. A number of analyses, particularly those studying host galaxy properties, perform a sequential regression which biases results when the regression covariates are correlated. Using a toy model of the problem, we calculated closed-form solutions for the size of these biases. We also used simulations based on publicly available Type Ia supernova data sets to measure the size of this bias in this context and provided correction terms.

Finally, in Chapter 6, we presented two new deep learning models of Type Ia supernova spectroscopy. These models extend the ``twins embedding" models presented in \citet{boone_twins_2020a} and \citet{boone_twins_2020b} beyond the narrow phase range in their original presentation. The first model, \texttt{spec2embed}, accurately predicts a supernova's location in the twins embedding space from a single spectrum from $-10$ to $+40$ days after maximum brightness, allowing us to standardize supernova brightnesses, and thus determine supernova distances, with comparable precision to the \citet{boone_twins_2020b} result using a wide range of spectra. The second model, \texttt{embed2spec}, allows us to use a forward-modeling approach to fitting observations to the twins embedding space, as it predicts a model spectral energy distribution given its phase and location in the twins embedding space.

There are a number of paths this research can progress. The \texttt{spec2embed} model can be used with existing instruments to precisely measure distances to nearby Type Ia supernovae, making possible measurements of growth of structure through observations of peculiar velocities, or to help constrain measurements of the Hubble constant. Combining the work of Chapters 4 and 6 would also be of immediate interest; the kernel density estimate technique applied to the SALT2 and SNEMO latent parameter spaces in Chapter 4 can easily be extended to the embedding space studied in Chapter 6, creating a fully generative, non-linear empirical model of Type Ia supernova spectral time-series data. This new generative model could be compared to the linear models of differing dimensionality studied in Chapter 4 as a means of quantitatively addressing the open question of the proper parametrization or description of Type Ia supernovae for cosmology. This model can also be used to generate low-resolution, noisy spectra (as we did in Chapter 3) or even photometric light curves, to see how well this information can be extracted from different types of measurements. The success of the \texttt{spec2embed} model in predicting the embedding coordinate across phases already suggests that the information content determining the absolute magnitude of a Type Ia supernova may be able to be extracted from data collected away from maximum brightness.

Upcoming surveys, like those that will be a part of the Rubin Observatory and Roman Space Telescope, will provide unprecedented numbers of supernovae, so many so that systematic biases will far outpace statistical uncertainties. Developing tools to identify, quantify, and mitigate these biases, as we have begun to do here, will be crucial in ensuring the accuracy of these cosmological measurements.
